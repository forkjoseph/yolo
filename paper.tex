\documentclass[12pt]{article}
% \documentclass{sig-alternate}
% \documentclass[conference]{IEEEtran}

% \usepackage{fixacm} % ACM format change
\usepackage[letterpaper,top=1in,bottom=1in,left=1in,right=1in]{geometry}
\usepackage[square,comma,numbers,sort&compress]{natbib}
\usepackage[breaklinks,colorlinks]{hyperref}
\usepackage[usenames,dvipsnames]{color}
\hypersetup{citecolor=blue,linkcolor=blue}
\usepackage{amsmath,amsopn,amssymb}
\usepackage{endnotes,microtype,xspace,graphicx,fancyvrb,multirow}
\usepackage{pgfplots,epsfig,caption,subcaption}
\let\labelindent\relax % stupid IEEE...
\usepackage{enumitem}
\usepackage{grffile}
\usepackage{comment}
% \usepackage{refcheck} % comment this out for final draft
\usepackage{verbatim}
% \usepackage{bigints}
\usepackage{textcomp}
\usepackage{url}
\usepackage[T1]{fontenc}
\usepackage{gensymb}
% \usepackage{authblk} % comment this out for IEEE
\usepackage{breakurl}

% \newcommand{\subparagraph}{}

%%% figs path
\graphicspath{ {figs/} }
\DeclareGraphicsExtensions{.pdf,.png,.jpg}
\setlist[enumerate]{itemsep=0mm}

\def\UrlBreaks{\do\/\do-}
\hypersetup{
     colorlinks   = true,
     citecolor    = blue,
     linkcolor    = magenta
}

\newcommand{\degrees}{$\!\!$\char23$\!$}
\renewcommand{\refname}{\centerline{References cited}}
% \titlespacing*{\subsection}{0pt}{1.1\baselineskip}{\parskip}
% this handles hanging indents for publications
\def\rrr#1\\{\par
\medskip\hbox{\vbox{\parindent=2em\hsize=6.12in
\hangindent=4em\hangafter=1#1}}}
\def\baselinestretch{1}

\newenvironment{ppl}{\fontfamily{cmr}\selectfont}{\par}
% \newcommand{\superscript}[1]{\ensuremath{^{\textrm{#1}}}}
% \def\sharedaffiliation{\end{tabular}\newline\begin{tabular}{c}}

\renewcommand{\ttdefault}{pxtt}
\newcommand{\URL}{\url}
\newcommand{\cc}[1]{\mbox{\smaller[0.5]\texttt{#1}}}

%\clubpenalty=10000
%\widowpenalty=10000

%\linespread{1.2}

\fvset{fontsize=\scriptsize,xleftmargin=8pt,numbers=left,numbersep=5pt}


\makeatletter
\def\PY@reset{\let\PY@it=\relax \let\PY@bf=\relax%
    \let\PY@ul=\relax \let\PY@tc=\relax%
    \let\PY@bc=\relax \let\PY@ff=\relax}
\def\PY@tok#1{\csname PY@tok@#1\endcsname}
\def\PY@toks#1+{\ifx\relax#1\empty\else%
    \PY@tok{#1}\expandafter\PY@toks\fi}
\def\PY@do#1{\PY@bc{\PY@tc{\PY@ul{%
    \PY@it{\PY@bf{\PY@ff{#1}}}}}}}
\def\PY#1#2{\PY@reset\PY@toks#1+\relax+\PY@do{#2}}

\expandafter\def\csname PY@tok@gd\endcsname{\def\PY@tc##1{\textcolor[rgb]{0.63,0.00,0.00}{##1}}}
\expandafter\def\csname PY@tok@gu\endcsname{\let\PY@bf=\textbf\def\PY@tc##1{\textcolor[rgb]{0.50,0.00,0.50}{##1}}}
\expandafter\def\csname PY@tok@gt\endcsname{\def\PY@tc##1{\textcolor[rgb]{0.00,0.27,0.87}{##1}}}
\expandafter\def\csname PY@tok@gs\endcsname{\let\PY@bf=\textbf}
\expandafter\def\csname PY@tok@gr\endcsname{\def\PY@tc##1{\textcolor[rgb]{1.00,0.00,0.00}{##1}}}
\expandafter\def\csname PY@tok@cm\endcsname{\let\PY@it=\textit\def\PY@tc##1{\textcolor[rgb]{0.25,0.50,0.50}{##1}}}
\expandafter\def\csname PY@tok@vg\endcsname{\def\PY@tc##1{\textcolor[rgb]{0.10,0.09,0.49}{##1}}}
\expandafter\def\csname PY@tok@m\endcsname{\def\PY@tc##1{\textcolor[rgb]{0.40,0.40,0.40}{##1}}}
\expandafter\def\csname PY@tok@mh\endcsname{\def\PY@tc##1{\textcolor[rgb]{0.40,0.40,0.40}{##1}}}
\expandafter\def\csname PY@tok@go\endcsname{\def\PY@tc##1{\textcolor[rgb]{0.53,0.53,0.53}{##1}}}
\expandafter\def\csname PY@tok@ge\endcsname{\let\PY@it=\textit}
\expandafter\def\csname PY@tok@vc\endcsname{\def\PY@tc##1{\textcolor[rgb]{0.10,0.09,0.49}{##1}}}
\expandafter\def\csname PY@tok@il\endcsname{\def\PY@tc##1{\textcolor[rgb]{0.40,0.40,0.40}{##1}}}
\expandafter\def\csname PY@tok@cs\endcsname{\let\PY@it=\textit\def\PY@tc##1{\textcolor[rgb]{0.25,0.50,0.50}{##1}}}
\expandafter\def\csname PY@tok@cp\endcsname{\def\PY@tc##1{\textcolor[rgb]{0.74,0.48,0.00}{##1}}}
\expandafter\def\csname PY@tok@gi\endcsname{\def\PY@tc##1{\textcolor[rgb]{0.00,0.63,0.00}{##1}}}
\expandafter\def\csname PY@tok@gh\endcsname{\let\PY@bf=\textbf\def\PY@tc##1{\textcolor[rgb]{0.00,0.00,0.50}{##1}}}
\expandafter\def\csname PY@tok@ni\endcsname{\let\PY@bf=\textbf\def\PY@tc##1{\textcolor[rgb]{0.60,0.60,0.60}{##1}}}
\expandafter\def\csname PY@tok@nl\endcsname{\def\PY@tc##1{\textcolor[rgb]{0.63,0.63,0.00}{##1}}}
\expandafter\def\csname PY@tok@nn\endcsname{\let\PY@bf=\textbf\def\PY@tc##1{\textcolor[rgb]{0.00,0.00,1.00}{##1}}}
\expandafter\def\csname PY@tok@no\endcsname{\def\PY@tc##1{\textcolor[rgb]{0.53,0.00,0.00}{##1}}}
\expandafter\def\csname PY@tok@na\endcsname{\def\PY@tc##1{\textcolor[rgb]{0.49,0.56,0.16}{##1}}}
\expandafter\def\csname PY@tok@nb\endcsname{\def\PY@tc##1{\textcolor[rgb]{0.00,0.50,0.00}{##1}}}
\expandafter\def\csname PY@tok@nc\endcsname{\let\PY@bf=\textbf\def\PY@tc##1{\textcolor[rgb]{0.00,0.00,1.00}{##1}}}
\expandafter\def\csname PY@tok@nd\endcsname{\def\PY@tc##1{\textcolor[rgb]{0.67,0.13,1.00}{##1}}}
\expandafter\def\csname PY@tok@ne\endcsname{\let\PY@bf=\textbf\def\PY@tc##1{\textcolor[rgb]{0.82,0.25,0.23}{##1}}}
\expandafter\def\csname PY@tok@nf\endcsname{\def\PY@tc##1{\textcolor[rgb]{0.00,0.00,1.00}{##1}}}
\expandafter\def\csname PY@tok@si\endcsname{\let\PY@bf=\textbf\def\PY@tc##1{\textcolor[rgb]{0.73,0.40,0.53}{##1}}}
\expandafter\def\csname PY@tok@s2\endcsname{\def\PY@tc##1{\textcolor[rgb]{0.73,0.13,0.13}{##1}}}
\expandafter\def\csname PY@tok@vi\endcsname{\def\PY@tc##1{\textcolor[rgb]{0.10,0.09,0.49}{##1}}}
\expandafter\def\csname PY@tok@nt\endcsname{\let\PY@bf=\textbf\def\PY@tc##1{\textcolor[rgb]{0.00,0.50,0.00}{##1}}}
\expandafter\def\csname PY@tok@nv\endcsname{\def\PY@tc##1{\textcolor[rgb]{0.10,0.09,0.49}{##1}}}
\expandafter\def\csname PY@tok@s1\endcsname{\def\PY@tc##1{\textcolor[rgb]{0.73,0.13,0.13}{##1}}}
\expandafter\def\csname PY@tok@sh\endcsname{\def\PY@tc##1{\textcolor[rgb]{0.73,0.13,0.13}{##1}}}
\expandafter\def\csname PY@tok@sc\endcsname{\def\PY@tc##1{\textcolor[rgb]{0.73,0.13,0.13}{##1}}}
\expandafter\def\csname PY@tok@sx\endcsname{\def\PY@tc##1{\textcolor[rgb]{0.00,0.50,0.00}{##1}}}
\expandafter\def\csname PY@tok@bp\endcsname{\def\PY@tc##1{\textcolor[rgb]{0.00,0.50,0.00}{##1}}}
\expandafter\def\csname PY@tok@c1\endcsname{\let\PY@it=\textit\def\PY@tc##1{\textcolor[rgb]{0.25,0.50,0.50}{##1}}}
\expandafter\def\csname PY@tok@kc\endcsname{\let\PY@bf=\textbf\def\PY@tc##1{\textcolor[rgb]{0.00,0.50,0.00}{##1}}}
\expandafter\def\csname PY@tok@c\endcsname{\let\PY@it=\textit\def\PY@tc##1{\textcolor[rgb]{0.25,0.50,0.50}{##1}}}
\expandafter\def\csname PY@tok@mf\endcsname{\def\PY@tc##1{\textcolor[rgb]{0.40,0.40,0.40}{##1}}}
\expandafter\def\csname PY@tok@err\endcsname{\def\PY@bc##1{\setlength{\fboxsep}{0pt}\fcolorbox[rgb]{1.00,0.00,0.00}{1,1,1}{\strut ##1}}}
\expandafter\def\csname PY@tok@kd\endcsname{\let\PY@bf=\textbf\def\PY@tc##1{\textcolor[rgb]{0.00,0.50,0.00}{##1}}}
\expandafter\def\csname PY@tok@ss\endcsname{\def\PY@tc##1{\textcolor[rgb]{0.10,0.09,0.49}{##1}}}
\expandafter\def\csname PY@tok@sr\endcsname{\def\PY@tc##1{\textcolor[rgb]{0.73,0.40,0.53}{##1}}}
\expandafter\def\csname PY@tok@mo\endcsname{\def\PY@tc##1{\textcolor[rgb]{0.40,0.40,0.40}{##1}}}
\expandafter\def\csname PY@tok@kn\endcsname{\let\PY@bf=\textbf\def\PY@tc##1{\textcolor[rgb]{0.00,0.50,0.00}{##1}}}
\expandafter\def\csname PY@tok@mi\endcsname{\def\PY@tc##1{\textcolor[rgb]{0.40,0.40,0.40}{##1}}}
\expandafter\def\csname PY@tok@gp\endcsname{\let\PY@bf=\textbf\def\PY@tc##1{\textcolor[rgb]{0.00,0.00,0.50}{##1}}}
\expandafter\def\csname PY@tok@o\endcsname{\def\PY@tc##1{\textcolor[rgb]{0.40,0.40,0.40}{##1}}}
\expandafter\def\csname PY@tok@kr\endcsname{\let\PY@bf=\textbf\def\PY@tc##1{\textcolor[rgb]{0.00,0.50,0.00}{##1}}}
\expandafter\def\csname PY@tok@s\endcsname{\def\PY@tc##1{\textcolor[rgb]{0.73,0.13,0.13}{##1}}}
\expandafter\def\csname PY@tok@kp\endcsname{\def\PY@tc##1{\textcolor[rgb]{0.00,0.50,0.00}{##1}}}
\expandafter\def\csname PY@tok@w\endcsname{\def\PY@tc##1{\textcolor[rgb]{0.73,0.73,0.73}{##1}}}
\expandafter\def\csname PY@tok@kt\endcsname{\def\PY@tc##1{\textcolor[rgb]{0.69,0.00,0.25}{##1}}}
\expandafter\def\csname PY@tok@ow\endcsname{\let\PY@bf=\textbf\def\PY@tc##1{\textcolor[rgb]{0.67,0.13,1.00}{##1}}}
\expandafter\def\csname PY@tok@sb\endcsname{\def\PY@tc##1{\textcolor[rgb]{0.73,0.13,0.13}{##1}}}
\expandafter\def\csname PY@tok@k\endcsname{\let\PY@bf=\textbf\def\PY@tc##1{\textcolor[rgb]{0.00,0.50,0.00}{##1}}}
\expandafter\def\csname PY@tok@se\endcsname{\let\PY@bf=\textbf\def\PY@tc##1{\textcolor[rgb]{0.73,0.40,0.13}{##1}}}
\expandafter\def\csname PY@tok@sd\endcsname{\let\PY@it=\textit\def\PY@tc##1{\textcolor[rgb]{0.73,0.13,0.13}{##1}}}

\def\PYZbs{\char`\\}
\def\PYZus{\char`\_}
\def\PYZob{\char`\{}
\def\PYZcb{\char`\}}
\def\PYZca{\char`\^}
\def\PYZam{\char`\&}
\def\PYZlt{\char`\<}
\def\PYZgt{\char`\>}
\def\PYZsh{\char`\#}
\def\PYZpc{\char`\%}
\def\PYZdl{\char`\$}
\def\PYZhy{\char`\-}
\def\PYZsq{\char`\'}
\def\PYZdq{\char`\"}
\def\PYZti{\char`\~}
% for compatibility with earlier versions
\def\PYZat{@}
\def\PYZlb{[}
\def\PYZrb{]}
\makeatother


\newcommand{\figrule}{\hrule width \hsize height .33pt}
\newcommand{\coderule}{\vspace{-0.4em}\figrule}

\setlength{\abovedisplayskip}{0pt}
\setlength{\abovedisplayshortskip}{0pt}
\setlength{\belowdisplayskip}{0pt}
\setlength{\belowdisplayshortskip}{0pt}
\setlength{\jot}{0pt}

\def\Snospace~{\S{}}
\renewcommand*\sectionautorefname{\Snospace}
\def\sectionautorefname{\Snospace}
\def\subsectionautorefname{\Snospace}
\def\subsubsectionautorefname{\Snospace}
\def\chapterautorefname{\Snospace}
%\renewcommand{\figurename}{Fig.}
%\def\figureautorefname{\figurename}
% \newcommand{\subfigureautorefname}{\figureautorefname}

%\numberwithin{equation}{section}
\newcommand{\yes}{Y}
\newcommand{\no}{}

% sema
\newcommand{\shl}{\ \cc{<}\cc{<}\ }
\newcommand{\shr}{\ \cc{>}\cc{>}\ }

\if 0
\renewcommand{\topfraction}{0.9}
\renewcommand{\dbltopfraction}{0.9}
\renewcommand{\bottomfraction}{0.8}
\renewcommand{\textfraction}{0.05}
\renewcommand{\floatpagefraction}{0.9}
\renewcommand{\dblfloatpagefraction}{0.9}
\setcounter{topnumber}{10}
\setcounter{bottomnumber}{10}
\setcounter{totalnumber}{10}
\setcounter{dbltopnumber}{10}
\fi

\newif\ifdraft\drafttrue
\newif\ifnotes\notestrue
\ifdraft\else\notesfalse\fi

% ref. http://en.wikibooks.org/wiki/LaTeX/Colors
\newcommand{\TK}[1]{\textcolor{LimeGreen}{TK: #1}}
\newcommand{\XXX}[1]{\textcolor{red}{XXX: #1}}
\newcommand{\TODO}[1]{\textcolor{Melon}{TODO: #1}}

% hide comments
% \renewcommand{\TK}[1]{\ignorespaces}
% \renewcommand{\XXX}[1]{\ignorespaces}
% \renewcommand{\TODO}[1]{\ignorespaces}

%% Ensure ligatures (e.g., ``fine official flag'') can be copy/pasted from PDF.
\input{glyphtounicode}
\pdfgentounicode=1

% \newcolumntype{R}[1]{>{\raggedleft\let\newline\\\arraybackslash\hspace{0pt}}p{#1}}

% include macros
\newcommand{\includepdf}[1]{
  \includegraphics[width=\columnwidth]{#1}
}
\newcommand{\includeplot}[1]{
  \resizebox{\columnwidth}{!}{\input{#1}}
}

% list
\newcommand{\squishlist}{
\begin{itemize}[noitemsep,nolistsep]
  \setlength{\itemsep}{-0pt}
}
\newcommand{\squishend}{
  \end{itemize}
}

\date{}

\begin{document}
% \title{Simple paper format}
\numberofauthors{1}

\author[*]{HyunJong JoSepH Lee}
\affil[*]{Georgia Tech} 
\renewcommand\Authands{ and }

%%% for IEEE
% \author{a1, a2, a3}
% \author{\IEEEauthorblockN{a1\IEEEauthorrefmark{1},
% a2\IEEEauthorrefmark{2} and
% a3\IEEEauthorrefmark{3}}\\
% \IEEEauthorblockA{\IEEEauthorrefmark{1}I1, a1@i1.edu}\\
% \IEEEauthorblockA{\IEEEauthorrefmark{2}I2, a2@i2.edu}\\
% \IEEEauthorblockA{\IEEEauthorrefmark{3}I3, a3@i3.edu}}


% \maketitle
\thispagestyle{empty}
\onecolumn
\setlength\parindent{0pt}
\nocite{*}

Q: {\it when two networks are available and given latency, b/w, data rate of each
network and transfer size, how to decide which one or both to use in respect to
weight in performance and data-usage? }
\newline

% problem justi.
When one network performs better than another with low or no cost for data
usage, selecting which network to use is straightforward. However, when one
network with a cost performs better than another with no cost, choosing which
network to select becomes difficult (performance vs. cost).
% statement
Our utility function policy takes account of performance (benefit) and data
consumption (cost) to transfer data so that an user can specify preference over
benefit and cost in 0 to 1 scale.
\newline

% net perf is resulted in time...!
First, our policy calculates performance of a network in seconds as following:
$$T(net)=T(size, b/w, latency) = \frac{size}{b/w}+latency$$ 
% in $\emptyset$ error or loss rate
where $size$ is amount of data to be transferred, $b/w$ and $latency$ are
bandwidth and end-to-end latency of a network. 
% Note that we use end-to-end latency of a network, instead of Round-Trip Time
% (RTT).
Details in calculating $b/w$ in respect to packet loss, error probabilities are
well-discussed in~\cite{macroscopic}.

% why do we need $$$ for D, instead of bytes?
Next, the cost of transferring data with size $S$ is expressed in dollar,
instead of bytes. That is because many network plans come with a budget for
% b/c "good" performance comes within only budgeted data.
regular latency or b/w (e.g., 2GB for LTE speed, 1TB for 15Mbps, etc) and after
the budget, either latency or b/w is throttled ({\it throttled stage}), although
one may claim that the amount an user pays includes not only budgeted data at
% sure it's unlimited, but it includes 2 stages!
regular latency or b/w but throttled stage.  Especially, when the goal of a
% when our goal is for performance, it's not good to step into throttled stage
policy is to maximize the benefit, efficiently scheduling within a boundary of
the budgeted data before throttled stage is crucial.
Compared to the data cost model proposed in IMP~\cite{imp}, which explicitly
states that stationary networks (e.g., Wifi) always outperforms in terms of data
rate, performance, and energy, our model estimates data cost associated
transferring a set of data size $S$ through a specific network $net$ with data
rate parameter to reflect a real world environment, where users generally pay
for X dollars for Y amount of data transferred at regular Z latency, as
following:
$$D(net)=D(size, data\_rate) = \frac{size}{data\_rate} = \text{cost in }\$ $$
Based on two metrics (performance and data cost), our scheduler selects via
which network(s) to transfer the data: when $T_{wifi} \leq T_{cell}$, always use
wifi, but when two metrics are contradicting such that $T_{wifi} > T_{cell}$ and
$D_{wifi} < D_{cell}$, we need additional heuristic called expected utility
value to conclude which network to use.
\nw

% why simple one does not work
% by decision theory (http://www.sfu.ca/~wainwrig/mpp/mpp-expectedutility.pdf)
Making a decision on which network to select is a derivative problem of decision
theory.
% risky function results in same value
A risky neutral function that has linear constant results in the expected
utility of benefit equal to the utility of its expected value. 
% thus it does not reflect "future" potential outcomes by taking this decision
In other words, a risky neutral function does not reflect potential ``future''
outcomes by taking current decision. For instance, a simple linear utility
function that takes performance (benefit) and data usage (cost) multiplied with
preference - 
$$U(net) = T_{net} + C_{t/\$}D_{net}$$
where $C_{t/\$}$ is a constant that an user willing to spend X dollars more to
save Y seconds - gives utility value equal to its expected value, thereby not
% linear one does not reflect preference
reflecting a consequence of taking current decision upon future outcomes and is
% thus it's bad
not suitable for our policy that tries to maximize benefit including future
outcomes by selecting a network.
A model, von Neumann-Morgenstern utility function shows that when a
% explain von Neumann function
decision-maker faces with risky (probabilistic) outcomes based on what he
chooses, he behaves to maximize the expected value of some utility function
defined over potential results in the future.
If we slightly modify the function to take a probability that a network will
give higher benefit (performance) based on historical distribution, we have:
$$U(net) = Prob_{ T_{net} \leq T_{others} }T_{net} + (1-Prob_{ T_{net} >
T_{others}})C_{t/\$}D_{net}$$
which now reflects ``future'' potential outcomes based on historical measures.
However, this model still lacks preference over a wealth. In our model, we
define a preference over a benefit from 0 to 1 scale and sum of all wealth to be
% C_t & C_d ==> what are they>??? 
1, i.e., $C_{perf} + C_{data} = 1$. Note that as a constant is closer to 1, an
user has strong preference over the metric (wealth). 
% why is better than sth simple??? 
% As discussed previously, risky neutral (linear function) model 

% budget to modify constant

% how to get params?
% b/w & latency -> from tcp
% size??
% 1) app tells the system
% 2) size distribution 
% 3) learn from characteristic of traffic by looking at traffic
%     --> is latency-sensitive
% A First Look at Traffic on Smartphones
% \nw
% B/w \& latency can be obtained from TCP stack. However, size of data transfer is




\newpage
%REF
\bibliographystyle{abbrv}
% \bibliographystyle{ieee}
% {\footnotesize{
  \bibliography{ref}
% }}
% \balancecolumns

\end{document}
